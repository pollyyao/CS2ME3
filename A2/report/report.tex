\documentclass[12pt]{article}

\usepackage{graphicx}
\usepackage{paralist}
\usepackage{listings}
\usepackage{hyperref}
\hypersetup{colorlinks=true,
    linkcolor=blue,
    citecolor=blue,
    filecolor=blue,
    urlcolor=blue,
    unicode=false}

\oddsidemargin 0mm
\evensidemargin 0mm
\textwidth 160mm
\textheight 200mm

\pagestyle {plain}
\pagenumbering{arabic}

\newcounter{stepnum}

\title{CS 2ME3 Assignment 2 Report}
\author{Name: Polly yao
             \and 
              MacId: yaos5}

\begin {document}
\maketitle

\lstset{language=Python, basicstyle=\tiny,breaklines=true,showspaces=false,showstringspaces=false,breakatwhitespace=true}

\def\thesection{\Alph{section}} 

\section{Code for pointADT.py} \label{pointSect}

\noindent \lstinputlisting{../src/pointADT.py}

\newpage

\section{Code for lineADT.py} \label{lineSect}

\noindent \lstinputlisting{../src/lineADT.py}

\newpage

\section{Code for circleADT.py} \label{cirlceSect}

\noindent \lstinputlisting{../src/circleADT.py}

\newpage

\section{Code for deque.py} \label{dequeSect}

\noindent \lstinputlisting{../src/deque.py}

\newpage

\section{Code for testCircleDeque.py} \label{testCircleDequeSect}

\noindent \lstinputlisting{../src/testCircleDeque.py}

\newpage

\section{Makefile} \label{MakefileSect}

\lstset{language=make}
\noindent \lstinputlisting{../Makefile}

\newpage

\section{Partner's circleADT Code} \label{PartnerCodeSect}

\lstset{language=Python}

\noindent \lstinputlisting{../src/srcPartner/circleADT.py}

\section*{Testing results of my files}

The results of testing my files came out successfully. Twenty test cases all completed in
0.067 seconds. However, I realized I have missed to include some important test cases.
Therefore, I decided to add some extra test cases just to see if my code really functions 
properly. First, I have added a test case for negative rotation, I can see my code functions
perfectly since if I rotate the point with the same radian both in positive and negative direction,
and the x and y coordinate will evetually go back to its original point. Second, I have added a 
test case for zero radians, and it also came out successfully, the point did not rotate at all. Third, 
i have added a test case for midpoint of a line of length zero, and the test was succusseful,
the midpoint of a line of length zero is the point itself. Overall 

\section*{Testing results of my files combined my partner's files}
The result of running my partner's files completed in 0.134 seconds. total of 20 test cases with 18 passed 
and 2 failed. testForce ans test intersect are two cases that failed. After seeing my partner's Force method,
I realized my force method is not appropriate, I should not create a counter method myself, and I should use
the functional language such as lambda. In the intersect method, I should have named my x and y componet 
as xc and yc as specified in the beginning of the pointADT module.

\section*{Discussion on the test results}
As stated previously,  my implementation for force and intersect method is different from 
my partner's. I think implement the correct specification and name is very important, In my
intersect method, I missed read the xc and yc coordinate which resulted in a non standardized
module. The formal mudule interface specification in this assignment makes the assignment more
standardize, not too much assumption need to stated. The advantages of using pyunit is that it
it could give the developer a earlier notice of what went wrong in the code instead of leading to a disaster
in the end. The pyunit method is also very simple to understand and gives the developer a time about
the efficeiency and perfomance of the code.




\end{document}