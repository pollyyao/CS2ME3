\documentclass[12pt]{article}

\usepackage{graphicx}
\usepackage{paralist}
\usepackage{listings}
\usepackage{hyperref}
\hypersetup{colorlinks=true,
    linkcolor=blue,
    citecolor=blue,
    filecolor=blue,
    urlcolor=blue,
    unicode=false}

\oddsidemargin 0mm
\evensidemargin 0mm
\textwidth 160mm
\textheight 200mm

\pagestyle {plain}
\pagenumbering{arabic}

\newcounter{stepnum}

\title{CS 2ME3 Assignment 1 Report}
\author{Name: Polly yao
             \and 
              MacId: yaos5}

\begin {document}
\maketitle

\lstset{language=Python, basicstyle=\tiny,breaklines=true,showspaces=false,showstringspaces=false,breakatwhitespace=true}

\def\thesection{\Alph{section}} 

\section{Code for CircleADT.py} \label{CircleSect}

\noindent \lstinputlisting{../src/CircleADT.py}

\newpage

\section{Code for Statistics.py} \label{StatisticsSect}

\noindent \lstinputlisting{../src/Statistics.py}

\newpage

\section{Code for testCircles.py} \label{testSect}

\noindent \lstinputlisting{../src/TestCircles.py}

\newpage

\section{Makefile} \label{MakefileSect}

\lstset{language=make}
\noindent \lstinputlisting{../Makefile}

\newpage

\section{Partner's CircleADT Code} \label{PartnerCodeSect}

\lstset{language=Python}

\noindent \lstinputlisting{../CircleADT.py}

\newpage

\section{Partner's Statistics Code} \label{PartnerCodeSect}

\lstset{language=Python}

\noindent \lstinputlisting{../Statistics.py}

\newpage

\section*{Testing results of my files and my partner's files}
The results of testing my CircleADT file and Statistics file came out successfully.
Each method is tested and passed by at least one test case. The rank method
is specifically tested with a test case which include duplicates, and this test case
has also successfully passed through. After finished testing my partner's file, 
several problemes  arise. First, the result came out different on the insideBox 
method between my partner and I. Second, my test cases does not work on my
partner's Statistics module, every method in his Statistis module assume to take
in four instances of CircleT seperately which means his methods need to take in 
four arguments, while all my methods in Statistics file only takes in one argument 
which is a list of instances of CircleT. Other than all the problems addressed 
previously, all other methods in his CircleADT.py works out perfectly, they have
passes through all my test cases and the result were exactly the same as mine.


\section*{Discussion on the test results}
As stated previously,  my result for insideBox method is different from the result 
of my partner's. His insideBox implemetation assumes a circle is inside a box if the 
circle's border is less than or equal to the border of the box, and my insideBox method 
assumes a circle is inside a box if the circle borader is only less than the border of the box.  
I consider both of our implementation for this method is appropriate, however we should 
all state our assumption in the doxygen comments.  Second, the conflicts occurs in the 
input arguments of Statistics module. My partner's module need to take in four instances 
of CircleT seperately and mine takes in a list of instances of CircleT, however his avaerage 
and standard deviation function works out perfectly after I changed my test cases to 
four seperate instance of CircleT. The result of my partner's rank function is still incorrect 
even after i changed the input into four seperate instance of CircleT; as I looked at his 
code in details, for every element in the given input list, the function finds its corresponding 
index in the sorted  list. However my implementation is exactly the opposite, for every 
element in the sorted list, my function finds its corresponding index in the given input list, 
and this is also what the instruction ask us to do. Finally, my partner did not take care of 
the situation where there is a tie in the list. From doing this exercise, I learned that 
documentation is very important for programs, especially when we have an assumption,
we should always state it in the doxygen comments.

\newpage
\section*{Discussion on how to handle the value of pi}
I have used the pi value from the standard math library of python. I have made this chocie 
because i think the pi value from the library will be more accurate instead of rounding the 
value myself. If we want to reuse this function in another procedure or class, the accurate 
result from previous step will greatly reduce the precision error in the final result. Using 
libraries(like numpy) has for sure improve the reliablilty since they are standardized, 
and their result will be very accurate,therefore it will be very reliable for other class 
or function to use it. The libraries could also make the program very productive since the 
algorithm implemented by ourselves might not be as efficiencent and quick as the one in 
the library. From this exersice, I have observed the differences of specifications or 
documentation would acutally cause differences in implementation. For examle, the 
specification in my Statistics.py is different from what my partner has in his Statistics.py 
since he takes in a different set of parameters.



\end{document}